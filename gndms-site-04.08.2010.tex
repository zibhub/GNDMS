\documentclass{article}
\usepackage{amssymb,amsmath}
\usepackage[mathletters]{ucs}
\usepackage[utf8x]{inputenc}
\usepackage{array}
% This is needed because raggedright in table elements redefines \\:
\newcommand{\PreserveBackslash}[1]{\let\temp=\\#1\let\\=\temp}
\let\PBS=\PreserveBackslash
\usepackage[breaklinks=true,unicode=true]{hyperref}
\setlength{\parindent}{0pt}
\setlength{\parskip}{6pt plus 2pt minus 1pt}
\setcounter{secnumdepth}{0}

\title{GNDMS Generation N Data Management System Documentation Bundle}
\author{Stefan Plantikow and Maik Jorra\\Zuse Institute Berlin (ZIB)}
\date{4.8.2010}


\begin{document}

\maketitle

\tableofcontents

\newpage


\section{Generation N Data Management System}

The Generation N Data Management System (\textbf{GNDMS}) is a set
of \href{http://www.globus.org}{Globus Toolkit 4} WSRF services and
associated tools for distributed grid data management based on
staging and co-scheduling. It abstracts from data sources via a
data integration layer and provides logical names, data transfers
via GridFTP, proper handling of GSI certificate delegation and
workspace management.

Besides data management functionality the implementation provides
components for remote logging, run-time reconfiguration,
persistence, and failover beyond what is available from
\href{http://www.globus.org}{Globus Toolkit 4}.

Originally, GNDMS was written and deployed for the data management
needs of the
\href{http://www.c3grid.de}{Collaborative Climate Community Data and Processing Grid (C3-Grid)}
and is now being used in the
\href{http://www.pt-grid.de}{Plasma-Technologie-Grid (PT-Grid)} as
part of the German \href{http://www.dgrid.de}{D-Grid} grid
computing initiative. Nevertheless, the implementation is flexible
and has been written for reuse by other grid projects with similiar
data management requirements. Core components may be of use to
developers of non-data management GT4 services as well.

\subsection{Impressum (DE)}

\begin{verbatim}
Konrad-Zuse-Zentrum fuer Informationstechnik Berlin
Takustrasse 7
D-14195 Berlin-Dahlem

GERMANY

Tel  +49 30 84185-0
Fax  +49 30 84185-125
Web  http://www.zib.de
\end{verbatim}

Das Konrad-Zuse-Zentrum fuer Informationstechnik Berlin ist eine Anstalt des oeffentlichen Rechts.

\subsubsection{Rechtliche Grundlagen}

\href{http://www.zib.de/General/Organization/gesetz/index.en.html}{Gesetz ueber das Zentrum fuer Informationstechnik (ZInfG)}

Trotz sorgfaeltiger inhaltlicher Kontrolle uebernehmen wir keine
Haftung fuer die Inhalte.

Sofern innerhalb unseres Internetangebotes die Moeglichkeit zur
Eingabe persoenlicher oder geschaeftlicher Daten (Emailadressen,
Namen, Anschriften usw.) besteht, so erfolgt die Preisgabe dieser
Daten seitens des Nutzers auf ausdruecklich freiwilliger Basis.

\subsection{Impressum (EN)}

\begin{verbatim}
Konrad-Zuse-Zentrum fuer Informationstechnik Berlin
Takustrasse 7
D-14195 Berlin-Dahlem

GERMANY

Tel  +49 30 84185-0
Fax  +49 30 84185-125
Web  http://www.zib.de
\end{verbatim}

ZIB was founded as an institution under public law.

\subsubsection{Legal Grounds}

\href{http://www.zib.de/General/Organization/gesetz/index.en.html}{Gesetz ueber das Zentrum fuer Informationstechnik (ZInfG)}

Although we carefully revise all texts we assume no liability for
the contents of external links for which solely the external
provider is responsible.

In case our Internet offer provides the option to enter personal or
business data (e.g. e-mail addresses, names, addresses), it is
deemed to be agreed that the user expressly voluntarily reveals
such data.

\section{GNDMS Installation Guide}

This is the Installation Guide for the
\href{%7B%7B%20page.root%20%7D%7Dindex.html}{Generation N Data Management System}.

\subsection{Prerequisites}

In order to build or install GNDMS, the following prerequisites
need to be fulfilled

\subsubsection{Prepare your local software installation}

Install the Java 2 SE Development Toolkit Version 1.6

Please Install the \href{http://java.sun.com}{Java} 2 SE
Development Toolkit Version 1.6.

For compiling the services, please make sure that \verb!$JAVA_HOME!
points to this version and that this is also the version that is in
your \verb!$PATH!. Naturally, this should be the same version than
the one you use(d) for building and running globus and ant.

Install Apache Ant 1.7

Please install \href{http://ant.apache.org}{Apache Ant} 1.7 and set
\verb!$ANT_HOME!, add it to your environment, and add
\verb!$ANT_HOME/bin! to your \verb!$PATH!

\textbf{NOTE} \emph{Using 1.8 might cause trouble on Linux, YMMV}

Install local UNIX software

In order to install GNDMS, please make sure you have installed the
following software

\begin{itemize}
\item
  openssl
\item
  curl
\item
  rsync
\end{itemize}
Additionally, it is expected that your UNIX provides the following
shell tools: hostname, which, bash

Install Globus Toolit 4.0.8

Please download and make a full installation of
\href{http://www.globus.org/toolkit/downloads/4.0.8/}{Globus Toolkit 4.0.8}

\textbf{NOTE}
\emph{To be precise, GNDMS doesn't use CAS and RLS, everything else needs to be there. However, due to the way the GT4 build system works, we suggest you just install everything.}

\textbf{NOTE}
\emph{If you want to cut down the build, try \texttt{./configure   --prefix=/opt/gt-4.0.8 --with-flavor=gcc32dbg --disable-rls   --disable-tests --disable-wstests --disable-drs} (or \texttt{--flavor=gcc64dbg} on 64-Bit Linux or Mac OS X)}

\begin{itemize}
\item
  Setup working host and user certificates (You can build without)
\item
  Set \verb!$GLOBUS_LOCATION! and add it to your environment
\item
  Life gets easier by putting
  \verb!source   $GLOBUS_LOCATION/etc/globus-user-env.sh! and
  \verb!source   $GLOBUS_LOCATION/etc/globus-devel-env.sh! in
  \verb!$HOME/.profile! when working with Globus Toolkit
\item
  We strongly suggest that you create a tarball of your fresh
  installation of globus toolkit for backup purposes. This allows you
  to rollback later and try again in case something goes wrong.
\end{itemize}
In the following it will be assumed that globus is run by the user
\verb!globus! which you will have created during the installation
of Globus Toolkit.

Optionally Configure Globus Toolkit Logging

This step is optional but highly recommended.

To configure the Globus Container to generate substantially more
log messages for easier debugging, please add

\begin{verbatim}
log4j.category.de.zib=DEBUG
\end{verbatim}
to \verb!$GLOBUS_LOCATION/container-log4j.properties!

For even more log information, please change the line that starts
with 

\verb!log4j.appender.A1.layout.ConversionPattern=! to

\begin{verbatim}
log4j.appender.A1.layout.ConversionPattern=%d{ISO8601} %-5p %c{2} [%t,%M:%L] <%x> %m%n
\end{verbatim}
in the same file.

\textbf{ATTENTION}
\emph{The default globus log file in \texttt{\$GLOBUS\_LOCATION/var/container.log} gets installed with very liberal file permisssions. You might want to \texttt{chmod 0640 \$GLOBUS\_LOCATION/var/container.log} for security reasons.}

Now it's time to start installing GNDMS.

\subsubsection{Preparation of GNDMS Software}

\textbf{ATTENTION}
\emph{The following steps need to be executed as the \texttt{globus} user that runs the servlet container of your installation of Globus Toolkit.}

Download and Unpack GNDMS

Download either an official GNDMS distribution package and unpack
it or get the current development version from github at:

  \verb!http://github.com/zibhub/GNDMS/downloads!

Please set \verb!$GNDMS_SOURCE! to the root directory of your GNDMS
distribution (The directory that contains \verb!Buildfile!) and add
\verb!$GNDMS_SOURCE/bin! to your \verb!$PATH!

Additionally, please set the following environment variables as specified below

\begin{itemize}
\item
  \verb!$GNDMS_SHARED! to \verb!$GLOBUS_LOCATION/etc/gndms_shared!
\item
  \verb!GNDMS_MONI_CONFIG! to \verb!$GNDMS_SHARED/monitor.properties!
\end{itemize}

After this step, there should be no further need to adjust your
environment. Please consult \verb!$GNDMS_SOURCE/example.profile!
for an example of a properly configured environment.

Optionally Install Apache buildr 1.4.1 locally

This step is optional.

GNDMS is built and installed using
\href{http:///buildr.apache.org}{Apache buildr}. A pre-packaged
version of buildr is included with GNDMS and can be executed by
running \verb!$GNDMS_SOURCE/bin/gndms-buildr!. However, if you
prever to install buildr locally, please

\begin{itemize}
\item
  Install Ruby 1.8
\item
  Install buildr by executing \verb!gem install buildr!
\end{itemize}
This guide assumes the usage of the pre-packaged version of
buildr.

\subsection{Installation and Deployment from Distribution Package}

This section describes the actual installation of the GNDMS
software into the Globus Container. It requires that your system
has been prepared as described in the previous section. Again the
following steps should be executed by the \verb!globus! user.

\begin{itemize}
\item
  Please enter \verb!$GNDMS_SOURCE! and exeucte
  \verb!gndms-buildr install-distribution!
\end{itemize}
This will

\begin{itemize}
\item
  Download and install required software dependencies into
  \verb!$GLOBUS_LOCATION/lib!.

  \textbf{Please consult \texttt{\$GNDMS\_SOURCE/doc/licensing} for details on licensing conditions of 3rd party software components used by the GNDMS package.}

\item
  Build API Documentation (Javadocs) in \verb!$GNDMS_SOURCE/doc/api!
\item
  and finally install the globus packages (gar-files)

\item
  (Re)start the globus container with
  \verb!globus-start-container-detached! and check
  \verb!$GLOBUS_LOCATION/var/container.log! If everything goes right
  and you enabled additional logging as described in the previous
  section, the output should contain output like

  {\small ============================================\\
  GNDMS RELEASE: Generation N Data Management System VERSION: 0.3\\
  ``Rob'' GNDMS BUILD: built-at: Wed Jul 21 11:14:01 +0200 2010\\
  built-by: mjorra@csr-pc35\\
  ============================================\\
  Container home directory is `/opt/gt-current'}

\end{itemize}
(In the case of an error, you may want to compare with a
\href{startup-log.txt}{full startup log}).

\begin{itemize}
\item
  \emph{After having checked succesful startup by looking at the logs},
  fix the permissions of database files by executing

  gndms-buildr fix-permissions

\end{itemize}
\textbf{ATTENTION}
\emph{Skipping this step may cause leaking of sensitive information to local UNIX users}

At this point the GNDMS software has been succesfully installed.
Next, we'll describe how it may be configured for actual use.

\subsection{Gridconfiguration of GNDMS Software}

GNDMS is configured via a builtin monitoring shell that accesses
and modifies the configuration in the database.

If you did a fresh installation, the monitoring shell will have
been enabled temporarily at this point and you may just proceed.
Otherwise you need to enable in manually as described in the
following section.

\subsubsection{Enabling the Monitoring Shell Manually}

To enable the monitor shell manually, after having startet the
globus container with deployed GNDMS at least once (as described in
the previous section), please edit \verb!$GNDMS_MONI_CONFIG! such
that \verb!monitor.enabled! is set to \verb!true! and either wait
until GNDMS picks up the new configuration or restart the globus
container.

The monitoring shell will be running now. You have nearly finished
the installation at this point. All that is left to do, is to
actually configure GNDMS for the chosen community grid platform.

\textbf{NOTE}
\emph{The shell is accessed via localhosts network interface and protected with a clear-text password only. Do not make the monitoring shell accessible via unsecure networks.}

\subsubsection{Disabling the Monitoring Shell}

To disable the monitor shell, please edit \verb!$GNDMS_MONI_CONFIG!
such that \emph{both} \verb!monitor.enabled! and
\verb!monitor.noShutdownIfRunning! are set to \verb!false!. Now,
either wait until the new configuration gets activated or just
restart the globus container manually.

\subsubsection{Configuring your Grid}

Currently, there are specialized build targets for the setup of
some D-Grid projects directly in the \verb!Buildfile!.

\textbf{PT-Grid Setup \& Configuration} : Edit
\verb!$GNDMS_SOURCE/scripts/ptgrid/setup-resource.sh! and execute
\verb!gndms-buildr ptgrid-setubdb!

\textbf{PT-Grid Quick Test} : Follow the setup instructions in the
testing section below and execute \verb!gndms-buildr ptgrid-test!

Additionally, please consult the documentation for the respective
community grid platform.

\textbf{NOTE}
\emph{In case of failure during setup, please execute \texttt{gndms-buildr kill-db} and try again.}

\subsubsection{Finalize installation}

Please edit
\verb!$GLOBUS_LOCATION/etc/gndms_shared/monitor.properties! and set
\verb!monitor.enabled=false! and
\verb!monitor.noShutdownIfRunning=false!. This will disable the
monitor shell after \verb!monitor.configRefreshCycle! ms (defaults
to 17 seconds). Alternatively, just restart the globus container.

\textbf{Congratulations}
\emph{At this point the installation is complete and you have a running installation of GNDMS.}

\subsection{Testing your installation}

The GNDMS contains a client application which tests some basic
functionality to ensure your setup is ready to use. In order to run
the test-client the following prerequisites must be satisfied:

\begin{itemize}
\item
  You must own a valid grid certificate,
\item
  have access to a grid-ftp-server, which accepts your certificate
  and offers write permission,
\item
  and of course a running you need a globus container that provides the
  GNDMS-services, has at least on subspace, and file-transfer
  enabled.
\end{itemize}
\subsubsection{About}

The test client simulates a standard GNDMS-use-case, it creates as
target slice, copies some files into the slice. Then it copies the
files back from the slice to some target directory and destroys the
slice.

This test should be executed as \textit{ordinary grid-user} not the
globus-user. (The only thing you need the globus-user for is to
edit the property-file, but more about that in the next section.)

\subsubsection{Setup}

For the scenario the following setup is required. On your grid-ftp
space create a directory and add some files, e.g.~using the
following bash command-line:

\begin{verbatim}
for i in $( seq 1 3 ); do \
    dd if=/dev/urandom bs=1024 count=1024 of=transfer_test$i.dat; done
\end{verbatim}

Additionally create a destination directory on the grid-ftp space.

The client reads its properties from a file:
\verb!$GNDMS_SOURCE/etc/sliceInOut.properties!. Now it's time to
edit this file. All properties whose values contain angle brackets
require attention. The file contains comments to every property and
hopefully explains itself. When you have finished the file must not
contain any angle-brackets, the client will complain if that's not
the case.

\subsubsection{Running the test client}

Once the setup is complete, load a grid-proxy using:

\begin{verbatim}
grid-proxy-init
\end{verbatim}
Now you can use buildr to fire up the client:

\begin{verbatim}
gndms-buildr gndms:gndmc:run-test
\end{verbatim}
(Or if provided: your grid specific test target) It takes quite
some time until the first output appears, be patient. After a
successful run your output start with:

\begin{verbatim}
Connected to GNDMS: Generation N Data Management System VERSION: 0.3-pre "Kylie++"
OK()
Creating slice
\end{verbatim}
(of course the version may differ) and end with:

\begin{verbatim}
Okay, all done. Cleaning up!
    * Destroying Slice
    * Destroying Delegate
Done.
\end{verbatim}
Click \href{test-output.txt}{here} to view the full output. If the
test runs successfully you should have identical files in your
grid-ftp source and destination directory, in that case
CONGRATULATIONS!! you have a working GNDMS installation, and can
provide data management service for your community.

\subsubsection{Trouble shooting}

\textbf{The client hangs after the }\verb|Copy gsiftp: ...|\textbf{
message.}

This can be a problem with your firewall configuration. It
happens when the control-channel can be established but the
data-channel is blocked. Please check your firewall setup
especially if the \verb!GLOBUS_TCP_PORT_RANGE! environment variable
is set correctly and is forwarded by the firewall.

\textbf{I'm getting a }\verb|GSSException: Defective credential detected|\textbf{ exception.}

This can have to reasons: your certificate-proxy maybe outdated
or doesn't exist or your CA directory isn't up to date. In the
first case just call \verb!grid-proxy-init! again, in the second
refer to the \verb!fetch-crl! section below.

\textbf{The file transfer throws an execption.}

 If the exeption looks something like
\begin{verbatim}
java.lang.IllegalStateException: File transfer from 
    gsiftp://some.foo.org:2811/tmp/srcDir to 
    gsiftp://more.bar.org/tmp/gndms/RW/f521ba10-a06a-11df-b70c-f2b2b7430fda failure 
    Server refused performing the request. ...`
\end{verbatim}
Or the client prints out infinite `Waiting for transfer to finish...`
messages and the destination directory contains a single empty file,
please ensure the both grid-ftp servers are running, accepting your
credential and can talk to each other. Best way to verify this is to
search the test-clients output for a line like:
 \begin{verbatim}
Copy gsiftp://some.foo.org:2811/tmp/srcDir ->
    gsiftp://more.bar.org/tmp/gndms/RW/f521ba10-a06a-11df-b70c-f2b2b7430fda
 \end{verbatim}
and try `globus-url-copy` with:
 \begin{verbatim}
globus-url-copy gsiftp://some.for.org:2811/tmp/srcDir/someFile \
    gsiftp://more.bar.org/tmp/gndms/RW/f521ba10-a06a-11df-b70c-f2b2b7430fda/targetFile
 \end{verbatim}
If you get an error message like "No route to host" or the like
ensure that the grid-ftp servers of both hosts are listening on the
right network device and that now firewall is blocking the connection.
If this hangs infinitely something with the data-channel setup is
wrong. Consult the grid-ftp documentation about the
\verb+--data-channel+ argument.

\subsection{Advanced Configuration}

\subsubsection{Remote Access to container.log}

To enable a select group of users to read the container.log from
outside, add their DNs to either

\verb!/etc/grid-security/gndms-support-stuff! or \\
\verb!$GLOBUS_LOCATION/etc/gndms_shared/gndms-support-stuff!.

Depending on your setup, you need to replace \verb!gndms! with your
subgrid name (\verb!ptgrid!, \verb!c3grid!, etc.) in these file
names.

To access the log, please load your user credentials (e.g. with
\verb!grid-proxy-init!) and run in \verb!$GNDMS_SOURCE!

\begin{verbatim}
`env URI="<URI>" ARGS="<ARGS>" gndms-buildr show-log`
\end{verbatim}
where \verb!<URI>! is the EPR of either a DSpace or a GORFX service
(see container.log startup section, looks like
\verb!https://$HOSTNAME:8443/wsrf/services/gndms/GORFX!) and
\verb!<ARGS>! are the arguments that need to be passed to the
actual show-log service maintenance call. Please use
\verb!env URI="<URI>" ARGS="help"! to obtain a synopsis of possible
parameters or leave it empty to retrieve
\verb!$GLOBUS_LOCATION/var/container.log! completely.

\subsubsection{Resetting the Database}

First, \textbf{shutdown the globus container}. Next, in
\verb!$GNDMS_SOURCE!, issue

gndms-buildr kill-db

This will delete your database.

\subsubsection{Inspecting the Database}

First, \textbf{shutdown the globus container}. Next, in
\verb!$GNDMS_SOURCE!, issue

gndms-buildr inspect-db

This will open a shell to the derby-ij tool for looking at the
internal database of GNDMS.

\subsubsection{Using the Monitor Shell}

Please consult the \href{/moni-guide}{monitor shell guide}

\subsection{Building GNDMS from Source}

Quick Rebuild

A quick full rebuild and reinstallation may be done by executing

\begin{verbatim}
gndms-buildr rebuild
\end{verbatim}
Regeneration of Javadocs

Manually delete \verb!$GLOBUS_LOCATION/doc/api!. Now regenerate the
javadocs by executing

\begin{verbatim}
gndms-buildr apidocs
\end{verbatim}
Building Manually from Scratch

\begin{verbatim}
gndms-buildr clean clean-services # Cleans everything
gndms-buildr artifcats            # Download all 3rd party components
gndms-buildr gndms:model:package  # Compile basic DAO classes
gndms-buildr package-stubs        # Compile service stubs
gndms-buildr gndms:infra:package  # Compile GNDMS framework
globus-stop-container-detached    # Ensure globus is shutdown
gndms-buildr install-deps         # Install dependencies
gndms-buildr package-DSpace       # Compile DSpace service
gndms-buildr deploy-DSpace        # Deploy DSpace
gndms-buildr package-GORFX        # Compile GORFX service
gndms-buildr deploy-GORFX         # Deploy GORFX
globus-start-container-detached   # Restart globus
gndms-buildr gndms:gndmc:package  # Build client
gndms-buildr apidocs              # Build Javadocs (gndms is excluded)
\end{verbatim}

\textbf{NOTE}
\emph{In order to get speedier builds, developers may set \texttt{\$GNDMS\_DEPS=link}. This will make \texttt{gndms-buildr install-deps} symlink dependencies to \texttt{\$GLOBUS\_LOCATION/lib} instead of copying them and therefore considerably eases trying out small changes to framework classes. However, when using this method, make sure that required symlinked jar files from \texttt{\$HOME/.m2/repository} and \texttt{\$GNDMS\_SOURCE/lib}and \texttt{\$GNDMS\_SOURCE/extra} are not deleted accidentally and remain readable for the globus user.}

\textbf{NOTE}
\emph{Once symlinks have been set up properly, developers may set \texttt{\$GNDMS\_DEPS=skip} to skip install-deps alltogether.}

\textbf{NOTE}
\emph{To even setup symlinks for the service jars, use the \texttt{gndms-buildr link-services} target.}

Packaging GNDMS

In case you want do distribute your own spin-of GNDMS, we suggest
you follow the procedure described below when making a release:

\begin{verbatim}
cd $GNDMS_SOURCE
vi Buildfile # Set VERSION_NUMBER and optionally VERSION_NAME
gndms-buildr release-build
git commit -m "Made a Release"
git tag gndms-release-ver
git push --tags origin master
find $GNDMS_SOURCE -type f -name '*.class' -exec rm '{}' \;
find $GNDMS_SOURCE -type d -name classes | xargs rm -fR
# Additionally delete 
#   */target 
#   name/gndms-name
#   dev-bin
#   test-data
# 
buildr apidocs
cd ..
mv $GNDMS_SOURCE $GNDMS_SOURCE/../gndms-release-ver
tar zcvf GNDMS-Releases/gndms-release-ver.tgz --exclude .git \
    --exclude *.ipr --exclude *.iml --exclude *.iws \
    --exclude *.DS_Store --exclude *._.DS_Store gndms-release-ver
mv $GNDMS_SOURCE/../gndms-release-ver $GNDMS_SOURCE
\end{verbatim}

Now, please upload the tarball and let the world know about it.

Problem Shooting Tips for Development Builds

\begin{itemize}
\item
  Do you need to regen the stubs?
  \verb!gndms-buildr clean-services   package-stubs! to the rescue.

\item
  Symlinks/copies of old jars in \verb!$GLOBUS_LOCATION/lib!.

  find \$GLOBUS\_LOCATION/lib -type l -name *.jar -exec rm -i \{\} ;

\end{itemize}
may help

\begin{itemize}
\item
  If you cant deploy (i.e. globus-start-container balks with one of
  those 40+-lines stacktraces) it's possible that introduce created
  an invalid jndi-config.xml which can happen during development but
  is easy enough to fix: Just make sure there are neither duplicate
  service nor resourceHome entries in any of the jndi-config.xml
  files

\item
  This build is not supposed to work on Microsoft Windows

\item
  If you get an error about a missing ``test/src'' directory simply
  mkdir -p test/src in the respective services' directory

\end{itemize}
Other common reasons for a failed container starts are invalid
credentials (hostkey/hostcert.pem) or outdated CRLs. In the latter
case, the script contained in
\verb!$GNDMS_SOURCE/contrib/fetch-crl! may help you. Execute
\verb!fetch-crl -o <grid-cert-dir>! with apropriate permissions
(Requries \verb!wget!).

\section{GNDMS Monitor Shell Guide}

This is the Monitor Shell Guide for the \href{../index.html}{Generation N Data Management System}.

If you have set up your environment as described in the
\href{/installation-guide/#prerequisites}{prerequisites section} of
the GNDMS Installation Guide you may use the GNDMS monitoring and
configuration shell to access a running instance of the GNDMS
software. This is a little servlet that allows the execution of
predefined actions or \href{http://groovy.codehaus.org}{Groovy 1.6}
script code inside the running globus container in order to
initialize and configure the database or peek inside the running
system for debugging purposes.

On most sites, the GNDMS Monitor Shell is only accessed once to
initialize the database during installation.

The GNDMS Monitor Shell is disabled by default and protected by a
randomly generated default password. If enabled, it opens a socket
on localhost for incoming connections. Please consider that
connections are \emph{unencrypted} before configuring it to be
accessible from an external network. Again, be aware that you can
truely execute arbitrary groovy code with globus user permissions
through this channel and therefore be cautious whenever using it.
You should enable it only on demand and always disable the service
after use.

To enable, edit \verb!$GNDMS_MONI_CONFIG! and set
\verb!monitor.enabled! to \verb!true!. Then either restart the
container or wait \verb!monitor.configRefreshCycl! ms (defaults to
17 seconds). After this period, the container will load your new
configuration and start the monitor shell automatically.

Alternatively, you may set \verb!$GNDMS_MONITOR_ENABLED! to
\verb!true! before starting the globus container to enable the
monitor.

There are two ways to use the monitor shell, the first allows the
execution of predefined actions, while the second runs arbitrary
Groovy code.

\subsection{Executing Actions}

To test the monitoring and configurations shell and retrieve a list
of all available actions, execute:

\begin{verbatim}
moni call help
\end{verbatim}
To call an action, execute:

\begin{verbatim}
moni call <Name of action> <Action Parameters or help>
\end{verbatim}
\subsection{Executing Groovy Code}

This mode of executions is based on http sessions.

\verb!moni init! creates a new session (Default session timeout is
22 mins). \verb!moni open repl foo! to create a new monitor named
``foo'' in the current session that accepts multiple commands
(\verb!repl! is the \emph{run mode} of the monitor. See below for a
list of possible run modes).

To use a previously opened monitor, open a second shell and
execute:

\begin{verbatim}
moni send foo $GNDMS_SOURCE/scrips/hello.groovy
\end{verbatim}
If you see \verb"Hello, World!" followed by \verb!null! in the
first shell you have succesfully enabled the monitor shell.

To close the connection named \verb!foo!, execute
\verb!moni close foo!. To destroy your session and close all named
connections, execute \verb!moni destroy!. To force the monitor to
reread the confuration, execute \verb!moni refresh!. To force a
restart even if the configuration has not been altered, execute
\verb!moni restart!.

\subsubsection{List of Supported Monitor Shell Run Modes}

\verb!SCRIPT! \emph{Default mode} : Accept one send command, do not
print result object.

\verb!REPL! : Accept many send commands, always print result
objects.

\verb!BATCH! : Accept many send commands, but do not print result
objects.

\verb!EVAL_SCRIPT! : Accept one send command, print result object.

\textbf{NOTE} Specifying the \verb!<mode>! in
\verb!moni open <mode> <connection-name>! is case-insensitive

\subsection{Appendix}

\subsubsection{Troubleshooting}

\begin{itemize}
\item
  If you don't get a connection, check
  \verb!$GLOBUS_LOCATION/var/container.log! and ensure that the GNDMS
  Monitor Shell has been started.

\item
  Make sure you have set up your environment as described in the
  \href{/installation-guide/#prerequisites}{prerequisites section} of
  the GNDMS Installation Guide.

\item
  If you execute moni and nothing happens you might just have
  forgotten an argument. Currently, moni is just a bunch of helper
  \verb!bash! scripts that call \verb!curl! and lack proper argument
  checking. If you do not provide \verb!moni send! with apropriate
  arguments, it may wait while attempting to read from stdin.

\item
  \verb!monitor.minConnections! should always be \textgreater{}= 2

\end{itemize}
\subsubsection{Tips for Script Developers}

Inside your own groovy classes, you should always print to
\verb!out! or \verb!err! which contain the current monitor's output
stream. Plain println only works correctly in the (outmost) script
scope or top-level functions.

\verb!out! and \verb!err! properties are added automatically to
\verb!Object.metaClass! when a monitor is instantiated. To enable
them, \verb!ExpandoMetaClass.enableGlobally()! is called first
which affects the semantics of Groovy.

Additional properties like resource homes and singleton resource
instances are made available using the same mechanism.

\section{GNDMS Architecture Primer}

This is the GNDMS Architecture Primer. It is developer-level
documentation that gives a general overview of the different layers
of GNDMS, how they inter-operate with Globus Toolkit, as well as
important concepts, components and classes. It is incomplete at
this point.

\subsection{The Layers of the Software Stack}

The running GNDMS software stack roughly looks like this:

\begin{verbatim}
< Service Clients ------------------------------------------------------ >
[ services: DSpace, GORFX ---------------------------------------------- ]    
[ gritserv-------------------------------------------------------------- ]
|                 [ infra ---------------------------------------------- ]
|                 [ logic ---------------------------------------------- ]
|                 |                                    < Configuration > |
|                 |                                    < Monitor Shell > |
|                 [ kit ------------------------------------------------ ]
|                 |                                                      |    
|                 |               [ model --------------------]          |
|                 |               |                           |          |    
|                 |               [ stuff -------- ]          |          |      
|                 |               |                |          |          |
[ Service Stubs ]-[ GT4 Container ]                [ Open-JPA ]-[ Groovy ]
|                                                  [ Derby    ]-[ Jetty  ]
|                                                                        |    
[ Additional External Libraries ---------------------------------------- ]
\end{verbatim}
Legend:

\begin{verbatim}
[ Module ]        # Software Module
                  # -- modules higher in the Stack
                  # -- depend on modules below them

< Tool   >        # Software Tool
                  # -- what is below is needed to run
\end{verbatim}
For the gndmc client, an additional layer is required:

\begin{verbatim}
< gndmc client --------------------------------------------------------- >
[ gndmc ---------------------------------------------------------------- ]
[ services: DSpace, GORFX ---------------------------------------------- ]    
..........................................................................
\end{verbatim}
The software stack for building, installing, deployment and
configuration of GNDMS roughly looks like this:

\begin{verbatim}
< buildr --------------------------------------------------------------- >
[ Build ---------------------------------------------------------------- ]
[ Buildr 1.4 ----------------------------------------------------------- ] 
|             [ introduce ------------ ]                                 |
[ JRuby 1.5 ] [ Ant ]                    < Configuration --------------- >    
[ Java 2 SDK 1.6 --------------------- ] < Monitor Shell --------------- >
< local UNIX tools ----------------------------------------------------- > 
\end{verbatim}
\subsection{Components}

\subsubsection{Component Categories}

There are several kinds of components that make up the GNDMS
service stack. They can be grouped into various categories

\textbf{Main} : GNDMS source code to be found below a top-level
diretory and directly built to a jar (Java)

\textbf{Service} : Located below \verb!services! and compiled with
introduce (XSD, WSDL, Java)

\textbf{Client} : Located below \verb!services! and compiled with
introduce (Part of service build process, consists of XSD, WSDL,
and Java)

\textbf{External} : downloaded from the Internet during
installation.

\textbf{Globus} : part of your local installation of Globus Toolkit
4.

\textbf{Build} : All code needed to build and install the software
(Ruby and Shell). Of primary relevance are the \verb!Buildfile! in
\verb!$GNDMS_SOURCE! and everything in
\verb!$GNDMS_SOURCE/buildr!.

\textbf{Config} : All code needed to configure GNDMS from the
outside (Shell and possibly Groovy)

\subsubsection{Components}

\textbf{Service Clients} \emph{(Client)} : Any program that
accesses GNDMS via WSRF.

\textbf{GORFX} \emph{(Service)} : The GORFX service provides the
negotiation and execution of data management activites (like
Staging and File Transfer) according to a co-scheduling protocol
(located below \verb!$GNDMS_SOURCE/services/GORFX!).

\textbf{DSpace} \emph{(Service)} : The DSpace service provides
management of storage resources (located below
\verb!$GNDMS_SOURCE/services/DSpace!).

\textbf{gritserv} \emph{(Main)} : Service-level code that is shared
between different grid services, like e.g.~XSD type conversion
(located below \verb!$GNDMS_SOURCE/gritserv!).

\textbf{infra} \emph{(Main)} : Main infrastructure code that ties
the service code to GNDMS \emph{main} classes. Configuration,
Database setup, Dependency injection (located below
\verb!$GNDMS_SOURCE/infra!).

\textbf{logic} \emph{(Main)} : All `Business' logic of GNDMS that
can be implemented outside of the actual service classes (located
below \verb!$GNDMS_SOURCE/logic!).

\textbf{Configuration} \emph{(Config)} : Configuration scripts
below \verb!$GNDMS_SOURCE! scripts are executing during
installation in order to configure a GNDMS instance for its purpose
in a given community grid (located below
\verb!$GNDMS_SOURCE/scrips!).

\textbf{Monitor Shell and Utilities} \emph{(Config)} : The monitor
shell (client) allows access to the monitor shell implemented
\textbf{kit} (located below \verb!$GNDMS_SOURCE/bin!).

\textbf{kit} \emph{(Main)} : Utility classes that depend on some
functionality from Globus Toolkit and/or \textbf{model}. Kit
contains the implementation of the GNDMS monitor shell and GridFTP
auxiliaries (located below \verb!$GNDMS_SOURCE/kit!).

\textbf{model} \emph{(Main)} : Database model classes (located
below \verb!$GNDMS_SOURCE/model!)

\textbf{stuff} \emph{(Main)} : Various utility classes (located
below \verb!$GNDMS_SOURCE/stuff!)

\textbf{Service Stubs} \emph{(Service)} : Stub code needed to
communicate with the Grid WSRF Services of GNDMS (located inside
services).

\textbf{GT4 Container} \emph{(Globus)} : The Globus Toolkit 4 WSRF
Service Container.

\textbf{Open JPA} \emph{(External)} : GNDMS uses Apache OpenJPA 2.0
as its Object Relational Mapper (ORM).

\textbf{Derby} \emph{(External)} : GNDMS uses Apache Derby 1.5 as
its underlying embedded database.

\textbf{Groovy} \emph{(External)} : GNDMS provides support to
access the system at runtime by means of executing groovy script
code via the monitor shell.

\textbf{Jetty} \emph{(External)} : The monitor shell is implemented
atop a stripped-down version of jetty.

\textbf{Additional External Libraries} \emph{(External)} : GNDMS
uses a large selection of 3rd party libraries. Please either
consult the \verb!Buildfile! or
\verb!$GNDMS_SOURCE/lib/gndms-depencies[.xml]! (post-install) to
find out more details. Consult \verb!$GNDMS_SOURCE/doc/licensing!
for licensing conditions of 3rd party components.

\textbf{Build} \emph{(Build)} : Build scripts are written in ruby
and placed in \verb!$GNDMS_SOURCE/Buildfile! and
\verb!$GNDMS_SOURCE/buildr/*!.

\textbf{Buildr 1.4} \emph{(Build)} : GNDMS relies on Apache Buildr
for build, installation, and deployment.

\textbf{introduce} \emph{(Build)} : The Introduce Tool from the
CAGrid project was used to generate service skeletons below
\verb!$GNDMS_SOURCE/services!.

\textbf{JRuby 1.5} \emph{(Build)} : \textbf{Buildr} needs this.

\textbf{Java 2 SDK 1.6} \emph{(All)} : GNDMS has been written in
Java.

\textbf{Documentation} : Documentation is generated using Javadoc
and Jekyll (has been installed in the included JRuby distribution)

GNDMS distribution packages contain a version of JRuby with
preinstalled buildr and Jekyll. This is not a part of GNDMS (You
could always fallback to your local installation of these tools)
and provided for convenience only.

\subsection{Suggested Code Walkthrough}

\begin{itemize}
\item
  Read the available documentation before entering the code, it will
  give you a rough idea of how everything is connected
\item
  Get to know the model classes
\item
  Read the action framework (Everything that inherits from
  de.zib.gndms.logic.action.Action)
\item
  Checkout \verb!infra/src/de/zib/gndms/infra/system/EMTools.java! to
  understand how actions and the database are connected
\item
  Checkout \verb!Ext*ResourceHome! in \textbf{DSpace} to see how
  resources are persisted.
\item
  Read the \verb!*ServiceImpl! classes to see the actual workflow
  that is triggered by incoming requests. Follow down to code in
  \textbf{logic} as you see fit.
\item
  Read \verb!infra/src/de/zib/gndms/infra/system/GNDMSystem.java! and
\item
  Read \verb!infra/src/de/zib/gndms/infra/system/GNDMSystemDirectory.java!
  to understand how GNDMS is bootstrapped and wired
\item
  The monitor is in kit in case you need to touch it
\end{itemize}

\section{GNDMS Developer Guide}

This is the Developer Guide for the
\href{../index.html}{Generation N Data Management System}. It is
far from complete. It currently contains various tidbits copied
together from different Wikis. YMMV. Use the source, luke!

\subsection{Writing Webservice Clients}

\subsubsection{Setup a Development Environment}

\begin{itemize}
\item
  Install GNDMS as described in the
  \href{../installation-guide}{installation guide}
\item
  Use \verb!gndms-buildr idea! or \verb!gndms-buildr eclipse! to
  generate template IDEA or eclipse projects.
\item
  You might need to add \verb!$GLOBUS_LOCATION/lib/*.jar!
\item
  Skip \verb!gndms-*.jar!, but
\item
  include \verb!gndms-*-service.jar! and \verb!gndms-*-client.jar!
\end{itemize}
\subsection{Setup a Development Environment for Debugging}

\begin{itemize}
\item
  Ensure that the generated modules in your IDE setup compile to the
  same output path as buildr and that globus, buildr, and your IDE
  compile using the same JDK.

\item
  Edit your globus scripts such that Java is configured to enable
  remote debugging and set up a matching run target in your IDE.

\end{itemize}
\textbf{NOTE}
\emph{If the globus container is started with \texttt{-debug} it prints full stacktraces, otherwise not!}

\subsubsection{Writing a Web Service Client}

\begin{itemize}
\item
  Take a look at \verb!ProviderStageInClient!
\item
  Do not directly instantiate port types. Always use the associated
  \verb!PortTypeFooClient! classes to get port type instances.
\item
  If you really need to instantiate port types directly, ensure that
  the used axis engine is configured with the correct \verb!.wsdd!
  files. This work is done by \verb!PortTypeFooClient! classes if you
  use them.
\end{itemize}
\subsection{Notes on Certificate Delegation}

To use certificate delegation, two steps are necessary. First,
service security settings need to be changed. Second, client and
and service code need to be modified slightly to incorporate
support for certificate delegation.

\subsubsection{Security Descriptor Basics}

\textbf{NOTE} This is a very brief description of security
descriptors. More advanced configuration is possible, e.g.~service
method level A\&A.

The security descriptor (Short: \textbf{SD}) describes
authentication and authorization requirements of clients and WSRF
web services. The \textbf{SD} of a service is configured in the
\verb!service! section of the WSDD file.

\begin{verbatim}
<?xml version="1.0" encoding="UTF-8"?>
<deployment>
    <service>
         <parameter name="securityDescriptor" value="etc/serviceFoo-security-desc.xml" /> 
    </service>
</deployment>
\end{verbatim}

Client security descriptors are loaded directly in the client
software:

\begin{verbatim}
// Client security descriptor file 
String CLIENT_DESC = ".../client-security-config.xml";
ClientSecurityDescriptor desc = new ClientSecurityDescriptor( CLIENT_DESC );
//Set descriptor on Stub 
( (Stub)port )._setProperty( Constants.CLIENT_DESCRIPTOR, desc )
\end{verbatim}

For more details, please consult the
\href{http://www.globus.org/toolkit/docs/development/4.1.2/security/security-secdesc.html}{documentation on security descriptors}.

\subsubsection{Authentication and Authorization}

The following example shows how mandatory TLS encryption is
enforced with a security descriptor:


\begin{verbatim}
<?xml version="1.0" encoding="UTF-8"?>
<securityConfig xmlns="http://www.globus.org">
    <auth-method>
        <GSITransport>
            <protection-level>
                <privacy />
            </protection-level>
        </GSITransport>
    </auth-method>
</securityConfig>
\end{verbatim}

This setting must be made both on the server and the client.

For authorization, a gridmap file needs to be set:


\begin{verbatim}
<authz value="gridmap" />
\end{verbatim}

This enables use of the system wide gridmap-file. To use a service
specific gridmap file, please add:


\begin{verbatim}
<gridmap value="etc/gndms_shared/grid-mapfile" />
\end{verbatim}

Finally, it is necessary to configure (unless you are using JAAS):

\begin{verbatim}
<run-as>
   <system-identity />
</run-as>
\end{verbatim}

Below is a complete example:

\begin{verbatim}
<?xml version="1.0" encoding="UTF-8"?>
<securityConfig xmlns="http://www.globus.org">
    <authz value="gridmap" />
    <gridmap value="etc/c3grid_shared/grid-mapfile" />
    <auth-method>
        <GSITransport>
            <protection-level>
                <privacy />
            </protection-level>
        </GSITransport>
    </auth-method>
    <run-as>
        <system-identity />
    </run-as>
</securityConfig>      
\end{verbatim}

\subsubsection{Client-Side Delegation}

This section described delegation from the viewpoint of the client.
The client uses the Delegation Service to retrieve the Certificate
Chain. This is used to generate a proxy certificate which is sent
to the delegation service in order to obtain an EPR for the proxy.
This EPR may be passed when accessing resources directly or is sent
to factory methods during resource instantiation.


// path to the file containing the proxy cert String proxyFile =
\ldots{};

\begin{verbatim}
// uri of the delegation service
String delUri =  "http://somehost/wsrf/services/DelegationFactoryService"

// port type of our service acquired in the usual fashion
PortType port = ... ;


GlobusCredential credential = new GlobusCredential( proxyFile );

// Create security descriptor for the communication with the delegation service
// This descriptor is not the same we use to communicate with
// the actual service
ClientSecurityDescriptor desc = new ClientSecurityDescriptor();
org.ietf.jgss.GSSCredential gss = 
    new org.globus.gsi.gssapi.GlobusGSSCredentialImpl( credential, 
   org.ietf.jgss.GSSCredential.INITIATE_AND_ACCEPT );
desc.setGSSCredential( gss );
desc.setGSITransport( (Integer) Constants.SIGNATURE );
Util.registerTransport();
desc.setAuthz( NoAuthorization.getInstance() );

EndpointReferenceType delegEpr = 
    AddressingUtils.createEndpointReference( delUri, null );

// acquire cert chain 
X509Certificate[] certs = 
    DelegationUtil.getCertificateChainRP( delegEpr, desc );

if( certs == null  )
     throw new Exception( "No Certs received" );

// create delegate
int ttl = 600; // a time to life for the proxy in seconds 
// the boolean value can be ignored
EndpointReferenceType delegate = 
    DelegationUtil.delegate( delUri, credential, certs[0], ttl, true, desc );

// reuse credentials for this call
( (Stub) port )._setProperty( 
    org.globus.axis.gsi.GSIConstants.GSI_CREDENTIALS, gss ); 

// creates a new resource which uses the delegate, i.e. proxy cert
EndpointReferenceType epr = 
    ( (SomePortType) port ).createResource( delegate );
\end{verbatim}

\subsubsection{Server-Side Delegation}

On the server side, the EPR needs to be used to retrieve the proxy
certificate. Additionally, a \verb!DelegationListener! needs to be
registered to be informed about proxy state changes (Update,
Destroy).

Example service factory method that instantiates a resource:

\begin{verbatim}
public EndpointReferenceType createResource ( EndpointReferenceType delegate ) {
    SomeResource sr = new SomeResource( );
    sr.setDelegationEPR( delegate );
    ...
    return endPointRefOf( sr );
}
\end{verbatim}

The resource needs to be modified accordingly as well:


\begin{verbatim}
public class SomeResource implements Resource {

    SomeResourceHome home;
    GlobusCredential credential;

    public void refreshRegistration( final boolean forceRefresh ) {
        // do refreshing stuff if required
    }


    public void setCredential( final GlobusCredential cred ) {
        credential = cred;
    }


    public GlobusCredential getCredential( ) {
        return credential;
    }


    public void setDelegateEPR( final EndpointReferenceType epr ) {

        SomeDelegationListener list = 
        new SomeDelegationListener( getResourceKey(), home );
        try {
            // registers listener with the delegation service
            DelegationUtil.registerDelegationListener( epr, list );
        } catch ( DelegationException e ) {
            e.printStackTrace();
        }
    }

    // other service specific stuff here ...
 }
\end{verbatim}

The container will be calling \verb!get/setCredential! on the
listener interface. A simple default implementation follows:

\begin{verbatim}
public class SomeDelegationListener implements DelegationListener {

    private static Logger logger = Logger.getLogger( SomeDelegationListener.class );
    private String regristrationId;
    private ResourceKey resourceKey;
    private ResourceHome home;


    public SomeDelegationListener() {
    }


    public SomeDelegationListener( final ResourceKey resourceKey, 
final ResourceHome home ) {
        this.resourceKey = resourceKey;
        this.home = home;
    }


    public void setCredential( final GlobusCredential credential )
throws DelegationException {

         try {
           SomeCredibleResource res = 
           ( SomeCredibleResource ) home.find( resourceKey );
           res.setCredential( credential );
         } catch ( ResourceException e ) {
           logger.error( e );
         }
    }


    public void credentialDeleted() {
       // Can notify the resource
    }

    // getters and setters for the instance vars are omitted for the sake of shortness
    // ....
}
\end{verbatim}

The \verb!setCredential! method will be called at listener
registration time.

With these extensions, a resource has access to the credentials of
the user to which the proxy belongs.

\subsubsection{Using Delegation with Proxy Certificates}

Service Orchestration

The main purpose of certificate delegation is to allow a service to
call another service on behalf of the user. Let's assume
\verb!SomeService! is a client of \verb!AnotherService!. In the
following example \verb!AnotherService! is called by
\verb!SomeService! with the proxy credentials by first loading them
into the \verb!ClientDescriptor!:

\begin{verbatim}
AnotherPortType port = ...;
    ( (Stub) port )._setProperty( org.globus.wsrf.security.Constants.GSI_TRANSPORT,
                        org.globus.wsrf.security.Constants.ENCRYPTION );
            // SIGNATUR should also work
    org.ietf.jgss.GSSCredential gss = 
    new org.globus.gsi.gssapi.GlobusGSSCredentialImpl( credential,
            org.ietf.jgss.GSSCredential.INITIATE_AND_ACCEPT );
    ( (Stub) port )._setProperty( 
    org.globus.axis.gsi.GSIConstants.GSI_CREDENTIALS, gss );
    ( (Stub) port ).doAnotherThing();
\end{verbatim}

Now, in \verb!AnotherService!, the caller DN (\verb!null! in
anonymous communication) is obtainable by calling:

\begin{verbatim}
 org.globus.wsrf.security.SecurityManager.getManager().getCaller();
\end{verbatim}

This may be mapped to local UNIX users via the grid-map mechanism:

\begin{verbatim}
  org.globus.wsrf.security.SecurityManager.getManager( ).getLocalUsernames()
\end{verbatim}

Export Proxy Credentials to a File


\begin{verbatim}
public void storeCredential( Sting filename ) {
     try {
         File f = new File( filename );
         FileOutputStream fos = new FileOutputStream( f );
         GlobusGSSCredentialImpl crd = 
         new GlobusGSSCredentialImpl( credential, GSSCredential.ACCEPT_ONLY );
         fos.write( crd.export( ExtendedGSSCredential.IMPEXP_OPAQUE  ) );
         fos.close();
     } catch( Exception e ) {
         // an exception --- do something
     }
}
\end{verbatim}

The resulting file is structured as follows:

\begin{itemize}
\item
  Proxy certificate generated last
\item
  Private key of this certificate
\item
  Certificate chain in descending order
\end{itemize}
The exported proxy may be verified manually with openssl by first
splitting this file into the \verb!head! (containing everything but
the certificate chain) and the \verb!tail! (containing the
certificate chain) and setting \verb!$OPENSSL_ALLOW_PROXY_CERTS=1!.
Now execute:

\begin{verbatim}
openssl verify -CApath /etc/grid-security/certificates -CAfile tail head
\end{verbatim}
If everything is ok, \verb!openssl! will print

\begin{verbatim}
head: OK
\end{verbatim}
Otherwise a lengthy error message will be shown.

Another way to acompish this is to install the tool
\href{http://www.nikhef.nl/~janjust/proxy-verify/}{grid-proxy-verify},
which handles the proxy-file without the need of splitting. A look
at the source code is an interesting read concerning the details of
proxy verification with openssl.

\subsection{Contract Semantics}

This section describes the precise semantics of offer contracts in
\emph{GNDMS}. An \emph{offer} is an negotiable offer for the
execution of a data management task (like Staging, Transfer, and
Publishing). Offer contracts may specify requirements on execution
time, duration and location. Offers are negotiated between a client
and server in rounds until agreement is reached and a contract is
succesfully established.

Client and server roles are taken by different participants. For
example, a grid meta scheduler may be a client to a central data
management site that runs GNDMS, while the same site may be a
client to a data provider site in a different negotiation.

\subsubsection{Protocol}

The protocol consists of three steps.

\begin{itemize}
\item
  Client send an OfferRequest with desired task and requirements
\item
  Server replies with an offer contract that tries to match clients
  requirements.
\item
  Client either accepts the offer. In this case, the protocol is
  finished with the creation of a task resource that allows the
  client to monitor task execution and to fetch results. Otherwise,
  the client is free to discard the offer and redo the protocol with
  another site or another contract.

\begin{verbatim}
Client                      GORFXServer 
   |                              |
   |     createOffer( ORQ )*      |
   X----------------------------->|--+
   |                              |  |
   |                              |  | Estimation( ORQ )
   |                              |  |
   |    offerContract             |<-+
   |<-----------------------------X
   |       Accept                 |
   X----------------------------->|--+
   |                              |  |
   |                              |  | run task
   |                              |  |
   |        Result                |<-+
   |<-----------------------------X
   |                              |
   .                              .
\end{verbatim}
\end{itemize}

\subsubsection{Contract Structure}

A contract consists of

\begin{itemize}
\item
  An (optional) point in time called \verb!IfDecisionBefore!,
  \textbf{IDB} for short
\item
  An (optional) point in time called \verb!ExecutionLikelyUntil!,
  \textbf{ELU} for short
\item
  A point in time or an offet calles \verb!ResultValidUntil!,
  \textbf{RVU} for short, or \textbf{Delta-RVU}, respectively
\item
  An (optional) size estimation calles \verb!EstMaxSize!,
  \textbf{EMS} for short (upper bound on the number of bytes of
  result data)
\item
  An (optional) set of key-value apires called \verb!RequestInfo!,
  \textbf{RI} for short. \textbf{RI} may be used to pass additional
  information like remarks, warnings etc. From a middleware point of
  view, \textbf{RI} is \emph{not} part of the contract.
\end{itemize}
\subsubsection{Contract Semantics}

\textbf{Precise contract semantics} : If the Offer is accepted
before \textbf{IDB}, the task will be executed before \textbf{ELU}
with high probability. Results are made available until
\textbf{RVU} as long as they do not need more than \textbf{EMS}
bytes of storage.

If \textbf{IDB} is missing, it is interpreted as an
\emph{arbitrary, undefined point in the future}. This is currently
not supported by the software but specifiable according to the
underlying XSD schema.

If \textbf{ELU} is missing, it is interpreted as
\emph{arbitrary, unknown task execution duration}. This is
currently not supported by the software but specifiable according
to the underlying XSD schema.

\subsubsection{Generic Client Restrictions}

\textbf{IDB} is mandatory. All mandatory invariants need to be
fulfilled.

\subsubsection{C3-Grid Data Provider Server Restrictions}

Contract semantic variables are mapped to staging properties as
detailed below:

\begin{verbatim}
IDB = c3grid.StageFileRequest.Estimate.IfDecisionBefore, 
ELU = c3grid.StageFileRequest.Estimate.ExecutionLikelyUntil, 
RVU = c3grid.StageFileRequest.Estimate.ResultValidUntil, 
EMS = c3grid.StageFileRequest.Estimate.MaxSize, 
RI  = c3grid.StageFileRequest.Estimate.RequestInfo
\end{verbatim}
\begin{itemize}
\item
  Data providers must specify an \textbf{ELU} that must be identical
  to the \textbf{ELU} requested from the client
\item
  \textbf{IDB} and \textbf{RVU} may not be modified by the server. In
  the case of \textbf{RVU} \textless{} \textbf{ELU}, the client
  should discard the request.
\item
  Client \textbf{EMS} may be discarded or overwritten by the server
  in his offer
\item
  \textbf{RI} is filtered for keys
\end{itemize}
\textbf{Summary} : For staging requests to data providers, it is
sufficient to specify \textbf{ELU} in ms and \textbf{EMS} in bytes,
and to optionally include key-value data in \textbf{RI}

\subsubsection{Support for Missing Timing Estimates}

\subsubsection{Contract Invariants}

Depending on how the contract requested by the client looks like,
some invariants need to be fulfilled:

\textbf{ELU, RVU} requested : \textbf{IDB} \textless{} \textbf{ELU}
and \textbf{IDB} \textless{} \textbf{RVU} (Therefore in practice,
choose \textbf{IDB} \textless{} \textbf{ELU} \textless{}
\textbf{RVU})

\textbf{Delta-ELU, RVU} requested : \textbf{IDB} \textless{}
\textbf{RVU} (Therefore in practice, choose \textbf{IDB} +
\textbf{Delta-ELU} \textless{} \textbf{RVU})

\textbf{ELU, Delta-RVU} requested : \textbf{IDB} \textless{}
\textbf{ELU} (and \textbf{RVU} is \textbf{ELU} + \textbf{Delta-RVU}
and therefore \textbf{ELU} \textless{}= \textbf{RVU} always holds)

\textbf{Delta-ELU, Delta-RVU} requested : No invariants, it holds
that start time \textbf{ST} \textless{}= \textbf{IDB}, completion
time \textbf{CT} = \textbf{ELU} = \textbf{ST} + \textbf{Delta-ELU},
\textbf{RVU} = \textbf{CT} + \textbf{Delta-RVU} and therefore
always \textbf{ELU} \textless{}= \textbf{RVU}

It always holds that
\textbf{ST}\textless{}=\textbf{IDB}\textless{}=\textbf{CT}\textless{}=\textbf{ELU}.
If \textbf{Delta-RVU} was requested. Additionally always
\textbf{CT} \textless{}= \textbf{ELU} \textless{}= \textbf{RVU} is
true.

Clients need to honor all invariants. Servers need to honor all
invariants which do not contain \textbf{RVU}. Servers may only
modify \textbf{ELU}.

\textbf{Contract} : GNDMS negotiates \emph{contracts} with clients
about task execution. A contract specifies what is to be done, and
optionally when and where it is to be done by GNDMS on behalf of
the client. \emph{Contracts} are accepted \emph{Offers}.

\textbf{Data Provider} : In \href{http://www.c3grid.de}{C3-Grid},
data providers are sites that run GNDMS with the Staging Plugin in
order to grant access to their local climate data archives.

\textbf{DMS} : Often used for the (or a) central data management
coordination site of a community grid.

\textbf{DMS-Publish} : In \href{http://www.c3grid.de}{C3-Grid},
load balancing publish to a dedicated storage server.

\textbf{DMS-Staging} : In \href{http://www.c3grid.de}{C3-Grid},
indirection of a staging request to a matching data provider or
cache.

\textbf{DSpace} : Workspace management service of GNDMS. Each
DSpace is structured into a set of \emph{subspaces} (Logical
stroage group). Each subspace consists of multiple \emph{slices}
(Non-hierachical container of files).

\textbf{GNDMS} : Generation N Data Management System. A data
management solution for community grids based on the
\href{http://www.globus.orrg}{Globus Toolkit 4 Middleware}.

\textbf{GORFX} (aka \textbf{G}eneric \textbf{O}ffer
\textbf{R}equest \textbf{F}actory \textbf{X}) : Service for the
negotation of data management task execution.

\textbf{Offer} : Cf. \emph{Contract}, offers are the subjects of
contract negotiation, Offers are not-yet accepted contracts.

\textbf{Offer Request} : Description of a task and required
\emph{offer} constraints.

\textbf{Publish} : In \href{http://www.c3grid.de}{C3-Grid},
publishing of intermediary results

\textbf{Publish-Host} : Host that provides storage resources for
\emph{Publish}. Needs to run \emph{DSpace} and \emph{GORFX}
configured for support of the \emph{Publish} task.

\textbf{Staging} : In \href{http://www.c3grid.de}{C3-Grid}, import
of climate data from external archives into the data management
infrastructure of the community grid.


\section{License}

GNDMS is distributed under the Apache License 2.0

Please consult the file LICENSE and everything below
`doc/licensing' in your release tarball for details.

Copyright 2008--2010 Zuse Institut Berlin (ZIB)

Licensed under the Apache License, Version 2.0 (the ``License'');
you may not use this file except in compliance with the License.
You may obtain a copy of the License at

\begin{verbatim}
   http://www.apache.org/licenses/LICENSE-2.0
\end{verbatim}
Unless required by applicable law or agreed to in writing, software
distributed under the License is distributed on an ``AS IS'' BASIS,
WITHOUT WARRANTIES OR CONDITIONS OF ANY KIND, either express or
implied. See the License for the specific language governing
permissions and limitations under the License.

\begin{verbatim}
                             Apache License
                       Version 2.0, January 2004
                    http://www.apache.org/licenses/
\end{verbatim}
TERMS AND CONDITIONS FOR USE, REPRODUCTION, AND DISTRIBUTION

  Definitions.

  ``License'' shall mean the terms and conditions for use,
  reproduction, and distribution as defined by Sections 1 through 9
  of this document.

  ``Licensor'' shall mean the copyright owner or entity authorized by
  the copyright owner that is granting the License.

  ``Legal Entity'' shall mean the union of the acting entity and all
  other entities that control, are controlled by, or are under common
  control with that entity. For the purposes of this definition,
  ``control'' means (i) the power, direct or indirect, to cause the
  direction or management of such entity, whether by contract or
  otherwise, or (ii) ownership of fifty percent (50\%) or more of the
  outstanding shares, or (iii) beneficial ownership of such entity.

  ``You'' (or ``Your'') shall mean an individual or Legal Entity
  exercising permissions granted by this License.

  ``Source'' form shall mean the preferred form for making
  modifications, including but not limited to software source code,
  documentation source, and configuration files.

  ``Object'' form shall mean any form resulting from mechanical
  transformation or translation of a Source form, including but not
  limited to compiled object code, generated documentation, and
  conversions to other media types.

  ``Work'' shall mean the work of authorship, whether in Source or
  Object form, made available under the License, as indicated by a
  copyright notice that is included in or attached to the work (an
  example is provided in the Appendix below).

  ``Derivative Works'' shall mean any work, whether in Source or
  Object form, that is based on (or derived from) the Work and for
  which the editorial revisions, annotations, elaborations, or other
  modifications represent, as a whole, an original work of
  authorship. For the purposes of this License, Derivative Works
  shall not include works that remain separable from, or merely link
  (or bind by name) to the interfaces of, the Work and Derivative
  Works thereof.

  ``Contribution'' shall mean any work of authorship, including the
  original version of the Work and any modifications or additions to
  that Work or Derivative Works thereof, that is intentionally
  submitted to Licensor for inclusion in the Work by the copyright
  owner or by an individual or Legal Entity authorized to submit on
  behalf of the copyright owner. For the purposes of this definition,
  ``submitted'' means any form of electronic, verbal, or written
  communication sent to the Licensor or its representatives,
  including but not limited to communication on electronic mailing
  lists, source code control systems, and issue tracking systems that
  are managed by, or on behalf of, the Licensor for the purpose of
  discussing and improving the Work, but excluding communication that
  is conspicuously marked or otherwise designated in writing by the
  copyright owner as ``Not a Contribution.''

  ``Contributor'' shall mean Licensor and any individual or Legal
  Entity on behalf of whom a Contribution has been received by
  Licensor and subsequently incorporated within the Work.

  Grant of Copyright License. 

  Subject to the terms and conditions of
  this License, each Contributor hereby grants to You a perpetual,
  worldwide, non-exclusive, no-charge, royalty-free, irrevocable
  copyright license to reproduce, prepare Derivative Works of,
  publicly display, publicly perform, sublicense, and distribute the
  Work and such Derivative Works in Source or Object form.

  Grant of Patent License. 

  Subject to the terms and conditions of
  this License, each Contributor hereby grants to You a perpetual,
  worldwide, non-exclusive, no-charge, royalty-free, irrevocable
  (except as stated in this section) patent license to make, have
  made, use, offer to sell, sell, import, and otherwise transfer the
  Work, where such license applies only to those patent claims
  licensable by such Contributor that are necessarily infringed by
  their Contribution(s) alone or by combination of their
  Contribution(s) with the Work to which such Contribution(s) was
  submitted. If You institute patent litigation against any entity
  (including a cross-claim or counterclaim in a lawsuit) alleging
  that the Work or a Contribution incorporated within the Work
  constitutes direct or contributory patent infringement, then any
  patent licenses granted to You under this License for that Work
  shall terminate as of the date such litigation is filed.

  Redistribution. 

  You may reproduce and distribute copies of the Work
  or Derivative Works thereof in any medium, with or without
  modifications, and in Source or Object form, provided that You meet
  the following conditions:

    You must give any other recipients of the Work or Derivative Works
    a copy of this License; and

    You must cause any modified files to carry prominent notices
    stating that You changed the files; and

    You must retain, in the Source form of any Derivative Works that
    You distribute, all copyright, patent, trademark, and attribution
    notices from the Source form of the Work, excluding those notices
    that do not pertain to any part of the Derivative Works; and

    If the Work includes a ``NOTICE'' text file as part of its
    distribution, then any Derivative Works that You distribute must
    include a readable copy of the attribution notices contained within
    such NOTICE file, excluding those notices that do not pertain to
    any part of the Derivative Works, in at least one of the following
    places: within a NOTICE text file distributed as part of the
    Derivative Works; within the Source form or documentation, if
    provided along with the Derivative Works; or, within a display
    generated by the Derivative Works, if and wherever such third-party
    notices normally appear. The contents of the NOTICE file are for
    informational purposes only and do not modify the License. You may
    add Your own attribution notices within Derivative Works that You
    distribute, alongside or as an addendum to the NOTICE text from the
    Work, provided that such additional attribution notices cannot be
    construed as modifying the License.

  You may add Your own copyright statement to Your modifications and
  may provide additional or different license terms and conditions
  for use, reproduction, or distribution of Your modifications, or
  for any such Derivative Works as a whole, provided Your use,
  reproduction, and distribution of the Work otherwise complies with
  the conditions stated in this License.

  Submission of Contributions. Unless You explicitly state otherwise,
  any Contribution intentionally submitted for inclusion in the Work
  by You to the Licensor shall be under the terms and conditions of
  this License, without any additional terms or conditions.
  Notwithstanding the above, nothing herein shall supersede or modify
  the terms of any separate license agreement you may have executed
  with Licensor regarding such Contributions.

  Trademarks. This License does not grant permission to use the trade
  names, trademarks, service marks, or product names of the Licensor,
  except as required for reasonable and customary use in describing
  the origin of the Work and reproducing the content of the NOTICE
  file.

  Disclaimer of Warranty. Unless required by applicable law or agreed
  to in writing, Licensor provides the Work (and each Contributor
  provides its Contributions) on an ``AS IS'' BASIS, WITHOUT
  WARRANTIES OR CONDITIONS OF ANY KIND, either express or implied,
  including, without limitation, any warranties or conditions of
  TITLE, NON-INFRINGEMENT, MERCHANTABILITY, or FITNESS FOR A
  PARTICULAR PURPOSE. You are solely responsible for determining the
  appropriateness of using or redistributing the Work and assume any
  risks associated with Your exercise of permissions under this
  License.

  Limitation of Liability. In no event and under no legal theory,
  whether in tort (including negligence), contract, or otherwise,
  unless required by applicable law (such as deliberate and grossly
  negligent acts) or agreed to in writing, shall any Contributor be
  liable to You for damages, including any direct, indirect, special,
  incidental, or consequential damages of any character arising as a
  result of this License or out of the use or inability to use the
  Work (including but not limited to damages for loss of goodwill,
  work stoppage, computer failure or malfunction, or any and all
  other commercial damages or losses), even if such Contributor has
  been advised of the possibility of such damages.

  Accepting Warranty or Additional Liability. While redistributing
  the Work or Derivative Works thereof, You may choose to offer, and
  charge a fee for, acceptance of support, warranty, indemnity, or
  other liability obligations and/or rights consistent with this
  License. However, in accepting such obligations, You may act only
  on Your own behalf and on Your sole responsibility, not on behalf
  of any other Contributor, and only if You agree to indemnify,
  defend, and hold each Contributor harmless for any liability
  incurred by, or claims asserted against, such Contributor by reason
  of your accepting any such warranty or additional liability.

END OF TERMS AND CONDITIONS
\end{document}
